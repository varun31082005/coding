% \iffalse
\let\negmedspace\undefined
\let\negthickspace\undefined
\documentclass[journal,12pt,twocolumn]{IEEEtran}
\usepackage{cite}
\usepackage{amsmath,amssymb,amsfonts,amsthm}
\usepackage{algorithmic}
\usepackage{graphicx}
\usepackage{textcomp}
\usepackage{xcolor}
\usepackage{txfonts}
\usepackage{listings}
\usepackage{enumitem}
\usepackage{mathtools}
\usepackage{gensymb}
\usepackage{comment}
\usepackage[breaklinks=true]{hyperref}
\usepackage{tkz-euclide}
\usepackage{listings}
\usepackage{gvv}
\def\inputGnumericTable{}
\usepackage[latin1]{inputenc}
\usepackage{color}
\usepackage{array}
\usepackage{longtable}
\usepackage{calc}
\usepackage{multirow}
\usepackage{hhline}
\usepackage{ifthen}
\usepackage{lscape}

\newtheorem{theorem}{Theorem}[section]
\newtheorem{problem}{Problem}
\newtheorem{proposition}{Proposition}[section]
\newtheorem{lemma}{Lemma}[section]
\newtheorem{corollary}[theorem]{Corollary}
\newtheorem{example}{Example}[section]
\newtheorem{definition}[problem]{Definition}
\newcommand{\BEQA}{\begin{eqnarray}}
\newcommand{\EEQA}{\end{eqnarray}}
\newcommand{\define}{\stackrel{\triangle}{=}}
\theoremstyle{remark}
\newtheorem{rem}{Remark}
\begin{document}

\bibliographystyle{IEEEtran}
\vspace{3cm}

\title{NCERT Discrete 11.9.3 -26}
\author{EE23BTECH11057 - Shakunaveti Sai Sri Ram Varun$^{}$% &lt;-this % stops a space
}
\maketitle
\newpage
\bigskip

\renewcommand{\thefigure}{\theenumi}
\renewcommand{\thetable}{\theenumi}
\vspace{2cm}
\textbf{Question: }
Insert two numbers between 3 and 81 so that the resulting sequence is G.P.\\
\textbf{Solution}:
A sequence is said to be in Geometric Progression when it is in the form of $ a, ar, ar^2, ar^3, ar^4..$\\
where $a$ is first term and $r$ is common ratio.\\
When there are four terms, the sequence becomes $ a, ar, ar^2, ar^3$\\
From the question given
\begin{align}
a = 3 \label{eq:1.11.9.3.26}\\
ar^3 = 81 \label{eq:2.11.9.3.26}
\end{align}
from $ \eqref{eq:1.11.9.3.26}$ and $ \eqref{eq:2.11.9.3.26}$ we get
\begin{align}
\notag r^3 = 27 \\
\notag r = 3
\end{align}
the sequence required is:
\begin{align}
3, 3 \times 3, 3 \times 3^2, 81
\end{align}
$ \therefore $ Required numbers are 9 and 27.

\end{document}
